%% LaTeX template for the science justification & technical
%% feasibility to be submitted as part of a TESS Guest Investigator
%% Program proposal. This template is based on the proposal template
%% used by the NuSTAR mission.
%%
%% TESS Guest Investigator Proposal Cycle 2 template
%% V1.0
%% 2017-08-04
%% V1.1
%% 2019-02-07

%%%%%%%%%%%%%%%%%%%%%%%%%%%
%%%%% DOCUMENT FORMAT %%%%%
%%%%%%%%%%%%%%%%%%%%%%%%%%%

%% The default font was chosen to be easily readable while allowing
%% sufficient material to be included.

%% Please note that the proposal will be printed on US Letter size paper,
%% 8.5 in x 11 in, and that formatting the text for other sizes will
%% generally cause layout problems and may result in text being cut
%% off near the edges. PLEASE DO NOT CHANGE THE 'LETTERPAPER' OPTION
%% IN THE DOCUMENTCLASS COMMAND.

%%%%%%%%%%%%%%%%%%%%%%%%%%%%%%%%%%%%%%%%%%%%%%
%%%%% Default format: 11pt single column %%%%%
%%%%%%%%%%%%%%%%%%%%%%%%%%%%%%%%%%%%%%%%%%%%%%

%% NOTE: NASA ROSES requires body font size to be no smaller than 15 
%% characters per inch (equivalent to Times Roman 12 point).
%%
%% Minimum margin size is 1 inch from top, bottom, and sides.

\documentclass[letterpaper,11pt]{article}

%%%%%%%%%%%%%%%%%%%%%%%%%%%%%%%%%%
%%%%% HOW TO INCLUDE FIGURES %%%%%
%%%%%%%%%%%%%%%%%%%%%%%%%%%%%%%%%%

%% Please see the ``Included packages'' section below.

%%%%%%%%%%%%%%%%%%%%%%%%%%%%%
%%%%% Included packages %%%%%
%%%%%%%%%%%%%%%%%%%%%%%%%%%%%

\usepackage{graphics,graphicx}
\usepackage[colorlinks]{hyperref}
\hypersetup{urlcolor=blue}
\usepackage{cleveref}

%% Feel free to modify the included packages list to use your
%% favorite packages. 

%% In the graphics and graphicx packages, Postscript and eps figures
%% can be included using the \includegraphics command. The graphics
%% package is part of standard LaTeX2e and provides a basic way of including a
%% figure. The graphicx package is not standard, but extends the
%% \includegraphics command to make it more user-friendly. If graphicx
%% is not available on your system please remove it from the list of
%% included packages above.  

%% Syntax:
%% In the graphics package:
%%
%% \begin{figure}
%% \includegraphics[llx,lly][urx,ury]{file}
%% \end{figure}
%%
%% where ll denotes 'lower left' and ur 'upper right' and the x and y
%% values are the coordinates of the PostScript bounding box in
%% points. There are 72 points in an inch.
%%
%% In the graphicx package:
%% 
%% \begin{figure}
%% \includegraphics[key=val,key=val,...]{file}
%% \end{figure}
%%
%% where some of the useful keys are: angle, width, height,
%% keepaspectratio (='true' or 'false') and scale. Bounding box values
%% can be given as [bb=llx lly urx ury].
%%
%% In either case you have to use LaTeX figure placement commands to
%% position the figure on the page; \includegraphics will not do
%% that. Both these commands also have other options that are listed
%% in the LaTeX manual (for the graphics package) and in 'The LaTeX
%% Graphics Companion' (for the graphicx package).



%%%%%%%%%%%%%%%%%%%%%%%%%%%
%%%%% Page dimensions %%%%%
%%%%%%%%%%%%%%%%%%%%%%%%%%%

\setlength{\textwidth}{6.5in} 
\setlength{\textheight}{9in}
\setlength{\topmargin}{-0.0625in} 
\setlength{\oddsidemargin}{0in}
\setlength{\evensidemargin}{0in} 
\setlength{\headheight}{0in}
\setlength{\headsep}{0in} 
\setlength{\hoffset}{0in}
\setlength{\voffset}{0in}



%%%%%%%%%%%%%%%%%%%%%%%%%%%%%%%%%%
%%%%% Section heading format %%%%%
%%%%%%%%%%%%%%%%%%%%%%%%%%%%%%%%%%

\makeatletter
\renewcommand{\section}{\@startsection%
{section}{1}{0mm}{-\baselineskip}%
{0.5\baselineskip}{\normalfont\Large\bfseries}}%
\makeatother

%%%%%%%%%%%%%%%%%%%%%%%%%%%%%%%%%%%%%
%%%%% Some Useful Abbreviations %%%%% 
%%%%%%%%%%%%%%%%%%%%%%%%%%%%%%%%%%%%%
\newcommand{\tess}{{\it TESS}}
\newcommand{\jwst}{{\it JWST}}
\newcommand{\kepler}{{\it Kepler}}
\newcommand{\ktwo}{{K2}}
\newcommand{\hst}{{\it HST}}
\newcommand{\msun}{$M_{\odot}$}
\newcommand{\rsun}{$R_{\odot}$}
\newcommand{\lsun}{$L_{\odot}$}
\newcommand{\re}{$R_{\oplus}$}
\newcommand{\me}{$M_{\oplus}$}
\newcommand{\rj}{$R_{\textrm{\scriptsize Jup}}$}
\newcommand{\mj}{$M_{\textrm{\scriptsize Jup}}$}
\newcommand{\ms}{m~s$^{-1}$}



%%%%%%%%%%%%%%%%%%%%%%%%%%%%%
%%%%% Start of document %%%%% 
%%%%%%%%%%%%%%%%%%%%%%%%%%%%%

\begin{document}
\pagestyle{plain}
\pagenumbering{arabic}


 
%%%%%%%%%%%%%%%%%%%%%%%%%%%%%
%%%%% Title of proposal %%%%% 
%%%%%%%%%%%%%%%%%%%%%%%%%%%%%

\begin{center} 
\bfseries\uppercase{%
%%
%% ENTER TITLE OF PROPOSAL BELOW THIS LINE
REPLACE THIS LINE WITH THE TITLE
%%
%%
}
\end{center}



%%%%%%%%%%%%%%%%%%%%%%%%%%%%%%%%%%%%%%%%%
%%%%% Body of science justification %%%%%
%%%%% and technical feasibility     %%%%%
%%%%%%%%%%%%%%%%%%%%%%%%%%%%%%%%%%%%%%%%%

\noindent{The recommended sections for a TESS GI proposal are shown below. Feel free to change section 
headings as necessary, but this is the suggested minimal information that should be included in the proposal. 
This Science/Technical section of the proposal is limited to 4 pages including figures and references.} 


\section{Introduction}

Summarize the problem being addressed and give an overview of how your investigation will help. 
Why \tess, why now?


\section{Scientific Justification}

Provide text and figures that justify the scientific need for \tess\ observations and analyses here. 
Justify your choice of cadence. When applicable, justify your choice of new 2 min cadence observations. 
If your program includes theoretical, simulation, or ground-based observing components (limited to no 
more than 30\% effort), describe why these efforts are critical in addition to \tess\ observations.


%% Trigger Criteria section:
%% comment this section out on proposals not asking for
%% Target of Opportunity (ToO) observations.

\subsection{({\it Only} For ToO Proposals) Trigger Criteria}

If the proposed investigation includes Targets of Opportunity (ToO's), describe also the circumstances 
under which a ToO is triggered, an estimated duration of the event(s), and an estimated probability for 
triggering the observations. Also discuss the potential science impact imposed by the delay in upload 
of the event due to \tess\ orbit/uplink constraints.


\section{Analysis Plan}

Discuss how you plan to analyze the \tess\ data. This includes the development of software tools.


\section{Technical Feasibility}

Provide text and figures showing that the proposed \tess\ investigations are feasible; consider the 
\tess\ survey strategy, target observability, and required signal-to-noise, etc. The \tess\ Science 
Support Center (\href{https://heasarc.gsfc.nasa.gov/docs/tess/}{TSSC}) makes several tools available to help estimate these quantities.


\section{Expected Impact}

Summarize the expected science return of the proposed investigations and the expected benefit 
to the community, including new data products and software tools to be made publicly available.


\section{Work Plan}

Provide a brief (1 paragraph) work plan that provides details on how the proposed effort will be carried out, including the allocation of effort amongst investigators. Investigators who are proposing to continue a program that was selected for funding in Cycle 1 must describe why additional funds are required in Cycle 2. All proposals requesting funds must also provide upon submission a bottom-line budget number in the provided field of the ARK submission form; this number should not be included in the body of the proposal.

\section{References}

List of references. References {\it are included} when considering the proposal 4 page limit.


\section{Target Table}

When necessary to justify your proposal, provide a list of targets using the below example as 
a template for format. This target table is designed to aid reviewers and need only provide a 
representative sample of the complete target list uploaded to RPS. Full target tables should be 
submitted electronically with the Phase-1 proposal. Please limit this target table to only 1 page. 
The table is {\it not included} in the 4 page limit of the Science/Technical section. Proposals including 
a target table may be up to 5 pages in length, but only the target table may appear on page 5.

\begin{center}
\begin{tabular}{ | c | c | c | c | c | c | c | }
\hline
TIC ID          &      Common      &     RA             &      Dec          &      TESS       &       Obj.        &      Comments \\       
                    &      Name           &     (deg)          &      (deg)        &      mag         &       Type       &                         \\     
\hline
\hline
388857263  &  Prox Cen          &  217.428793  &  -62.679592  &  7.36             &    M Dwarf    & 2 min cad., RV planet \\ \hline
                    &                           &                       &                      &                      &                     &                                     \\ \hline
                    &                           &                       &                      &                      &                     &                                     \\ \hline
                    &                           &                       &                      &                      &                     &                                     \\ \hline
                    &                           &                       &                      &                      &                     &                                     \\ \hline
\end{tabular}
\end{center}   

%%%%%%%%%%%%%%%%%%%%%%%%%%%
%%%%% End of document %%%%%
%%%%%%%%%%%%%%%%%%%%%%%%%%%

\end{document}

