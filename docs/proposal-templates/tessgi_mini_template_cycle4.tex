%% LaTeX template for the science justification & technical
%% feasibility to be submitted as part of a TESS Guest Investigator
%% Program proposal. This template is based on the proposal template
%% used by the NuSTAR mission.
%%
%% TESS Guest Investigator Proposal Cycle 4 template
%% V1.0
%% 2017-08-04
%% V1.1
%% 2019-02-07
%% V1.2
%% 2019-10-27
%% V1.3
%% 2020-11-02

%%%%%%%%%%%%%%%%%%%%%%%%%%%
%%%%% DOCUMENT FORMAT %%%%%
%%%%%%%%%%%%%%%%%%%%%%%%%%%

%% The default font was chosen to be easily readable while allowing
%% sufficient material to be included.

%% Please note that the proposal will be printed on US Letter size paper,
%% 8.5 in x 11 in, and that formatting the text for other sizes will
%% generally cause layout problems and may result in text being cut
%% off near the edges. PLEASE DO NOT CHANGE THE 'LETTERPAPER' OPTION
%% IN THE DOCUMENTCLASS COMMAND.

%%%%%%%%%%%%%%%%%%%%%%%%%%%%%%%%%%%%%%%%%%%%%%
%%%%% Default format: 12pt single column %%%%%
%%%%%%%%%%%%%%%%%%%%%%%%%%%%%%%%%%%%%%%%%%%%%%


%% Minimum margin size is 1 inch from top, bottom, and sides.
%% Font size: see NASA Guidebook for Proposers 
%%(https://www.hq.nasa.gov/office/procurement/nraguidebook/proposer2018.pdf).

\documentclass[letterpaper,12pt]{article}

%%%%%%%%%%%%%%%%%%%%%%%%%%%%%%%%%%
%%%%% HOW TO INCLUDE FIGURES %%%%%
%%%%%%%%%%%%%%%%%%%%%%%%%%%%%%%%%%

%% Please see the ``Included packages'' section below.

%%%%%%%%%%%%%%%%%%%%%%%%%%%%%
%%%%% Included packages %%%%%
%%%%%%%%%%%%%%%%%%%%%%%%%%%%%

\usepackage{graphics,graphicx}
\usepackage[colorlinks]{hyperref}
\hypersetup{urlcolor=blue}
\usepackage{cleveref}

%% Feel free to modify the included packages list to use your
%% favorite packages. 

%% In the graphics and graphicx packages, Postscript and eps figures
%% can be included using the \includegraphics command. The graphics
%% package is part of standard LaTeX2e and provides a basic way of including a
%% figure. The graphicx package is not standard, but extends the
%% \includegraphics command to make it more user-friendly. If graphicx
%% is not available on your system please remove it from the list of
%% included packages above.  

%% Syntax:
%% In the graphics package:
%%
%% \begin{figure}
%% \includegraphics[llx,lly][urx,ury]{file}
%% \end{figure}
%%
%% where ll denotes 'lower left' and ur 'upper right' and the x and y
%% values are the coordinates of the PostScript bounding box in
%% points. There are 72 points in an inch.
%%
%% In the graphicx package:
%% 
%% \begin{figure}
%% \includegraphics[key=val,key=val,...]{file}
%% \end{figure}
%%
%% where some of the useful keys are: angle, width, height,
%% keepaspectratio (='true' or 'false') and scale. Bounding box values
%% can be given as [bb=llx lly urx ury].
%%
%% In either case you have to use LaTeX figure placement commands to
%% position the figure on the page; \includegraphics will not do
%% that. Both these commands also have other options that are listed
%% in the LaTeX manual (for the graphics package) and in 'The LaTeX
%% Graphics Companion' (for the graphicx package).



%%%%%%%%%%%%%%%%%%%%%%%%%%%
%%%%% Page dimensions %%%%%
%%%%%%%%%%%%%%%%%%%%%%%%%%%

\setlength{\textwidth}{6.5in} 
\setlength{\textheight}{9in}
\setlength{\topmargin}{-0.0625in} 
\setlength{\oddsidemargin}{0in}
\setlength{\evensidemargin}{0in} 
\setlength{\headheight}{0in}
\setlength{\headsep}{0in} 
\setlength{\hoffset}{0in}
\setlength{\voffset}{0in}



%%%%%%%%%%%%%%%%%%%%%%%%%%%%%%%%%%
%%%%% Section heading format %%%%%
%%%%%%%%%%%%%%%%%%%%%%%%%%%%%%%%%%

\makeatletter
\renewcommand{\section}{\@startsection%
{section}{1}{0mm}{-\baselineskip}%
{0.5\baselineskip}{\normalfont\Large\bfseries}}%
\makeatother

%%%%%%%%%%%%%%%%%%%%%%%%%%%%%%%%%%%%%
%%%%% Some Useful Abbreviations %%%%% 
%%%%%%%%%%%%%%%%%%%%%%%%%%%%%%%%%%%%%
\newcommand{\tess}{{\it TESS}}
\newcommand{\jwst}{{\it JWST}}
\newcommand{\kepler}{{\it Kepler}}
\newcommand{\ktwo}{{K2}}
\newcommand{\hst}{{\it HST}}
\newcommand{\swift}{{\it Swift}}
\newcommand{\integral}{{\it INTEGRAL}}
\newcommand{\nustar}{{\it NuSTAR}}
\newcommand{\fermi}{{\it Fermi}}
\newcommand{\ms}{$M_{\odot}$}
\newcommand{\rs}{$R_{\odot}$}
\newcommand{\ls}{$L_{\odot}$}
\newcommand{\re}{$R_{\oplus}$}
\newcommand{\me}{$M_{\oplus}$}
\newcommand{\kms}{km~s$^{-1}$}
\newcommand{\fluxcgs}{ergs~s$^{-1}$~cm$^{-2}$}
\newcommand{\lumcgs}{ergs~s$^{-1}$}
\newcommand{\rj}{$R_{\textrm{\scriptsize Jup}}$}
\newcommand{\mj}{$M_{\textrm{\scriptsize Jup}}$}
\newcommand{\msec}{m~s$^{-1}$}


%%%%%%%%%%%%%%%%%%%%%%%%%%%%%
%%%%% Start of document %%%%% 
%%%%%%%%%%%%%%%%%%%%%%%%%%%%%

\begin{document}
\pagestyle{plain}
\pagenumbering{arabic}


 
%%%%%%%%%%%%%%%%%%%%%%%%%%%%%
%%%%% Title of proposal %%%%% 
%%%%%%%%%%%%%%%%%%%%%%%%%%%%%

\begin{center} 
\bfseries\uppercase{%
%%
%% ENTER TITLE OF PROPOSAL BELOW THIS LINE
REPLACE THIS LINE WITH YOUR MINI PROPOSAL TITLE
%%
%%
}
\end{center}



%%%%%%%%%%%%%%%%%%%%%%%%%%%%%%%%%%%%%%%%%
%%%%% Body of science justification %%%%%
%%%%% and technical feasibility     %%%%%
%%%%%%%%%%%%%%%%%%%%%%%%%%%%%%%%%%%%%%%%%


%%%%%%%%%%%%%%%%%%%%%%%%%%%%%%%%%%%%%%%%%
%%%%%%%%%%%%%%%%%%%%%%%%%%%%%%%%%%%%%%%%%
%%%%% The text below should be commented out before submitting your proposal %%%%% 
\noindent{The recommended sections for a \tess\ GI proposal are shown below. Feel free to change section 
headings as necessary, but this is the suggested minimal information that should be included in the proposal. 
This Science/Technical section of the proposal is limited to 2 pages. Figures are included in these page limits, but not references or a (sample) target table. \\

\noindent Note that the Phase-1 proposal review will be done in a dual-anonymous fashion and follow the guidelines listed below:
\begin{itemize}
    \item Proposals should eliminate language that identifies the proposers or institution, as discussed in the \href{https://science.nasa.gov/researchers/dual-anonymous-peer-review}{Guidelines for Anonymous Proposals}.
    \item PIs are required to upload a one-page Team Expertise (insert link) PDF through a separate upload when submitting the science justification into ARK/RPS.
    \item Proposals that do not follow these dual-anonymous guidelines may be returned without review.
\end{itemize}

\noindent Mini proposals are intended for requests for a small number of target slots and require minimal resources, up to 50 20-second cadence targets and 1,000 2-minute cadence targets. Proposals in this category are not eligible for funding.\\ 

\noindent Mini proposals cannot have Targets of Opportunity, a joint component with \hst, \swift, or \fermi, or have a ground-based component. 
 } 
%%%%% The text above should be commented out before submitting your proposal %%%%% 
%%%%%%%%%%%%%%%%%%%%%%%%%%%%%%%%%%%%%%%%%
%%%%%%%%%%%%%%%%%%%%%%%%%%%%%%%%%%%%%%%%%




\section{Scientific Justification and Perceived Impact}

Provide text and figures that justify the scientific need for new \tess\ observations and analyses here. In particular, justify your choice of new 2~min or 20~s cadence observations. If you will also be making use of the 10~min FFIs for your research, make it clear why the \tess\ FFI data are suitable for your science.


Summarize the expected science return of the proposed investigations and the expected benefit to the community.



\section{Analysis Plan and Technical Feasibility}

Discuss how you plan to analyze the \tess\ data. Provide text and figures showing that the proposed \tess\ investigations are feasible; consider the 
\tess\ survey strategy, target observability, and required signal-to-noise, etc. The \tess\ Science 
Support Center (\href{https://heasarc.gsfc.nasa.gov/docs/tess/}{TSSC}) makes several tools available to help estimate these quantities. 


\section{References}

List of references. References {\it are {\bf not} included} when considering the
proposal page limit. References in the text should be in the number format, and in the references list as:

[1] Person A, Person B, Person C, et al., 2016, ApJ 200, 231, 2\\
[2] Person D \& Person E, 1912, Nature 495, 452


\section{Target Table}

When necessary to justify your proposal, provide a list of targets using the below example as a template for format. This target table is designed to aid reviewers and need only provide a representative sample of the complete target list uploaded to RPS. Full target tables should be submitted electronically with the Phase-1 proposal. Please limit any target table included here to only 1 page. The table is not included in the page limit of the Science/Technical section. 


\begin{center}
\begin{tabular}{ | c | c | c | c | c | c | c | }
\hline
TIC ID          &      Common      &     RA             &      Dec          &      TESS       &       Obj.        &      Comments \\       
                    &      Name           &     (deg)          &      (deg)        &      mag         &       Type       &                         \\     
\hline
\hline
388857263  &  Prox Cen           &  217.428793  &  -62.679592  &  7.36             &    M Dwarf    & 2 min cad., RV planet \\ \hline
353622691  &  BL Lac               &   330.6803807    &   42.2777717    &   13.1  &   AGN            &    20 s cad.                                 \\ \hline
                    &                           &                       &                      &                      &                     &                                     \\ \hline
                    &                           &                       &                      &                      &                     &                                     \\ \hline
                    &                           &                       &                      &                      &                     &                                     \\ \hline
\end{tabular}
\end{center}   

%%%%%%%%%%%%%%%%%%%%%%%%%%%
%%%%% End of document %%%%%
%%%%%%%%%%%%%%%%%%%%%%%%%%%

\end{document}

