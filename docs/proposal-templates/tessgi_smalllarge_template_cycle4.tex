%% LaTeX template for the science justification & technical
%% feasibility to be submitted as part of a TESS Guest Investigator
%% Program proposal. This template is based on the proposal template
%% used by the NuSTAR mission.
%%
%% TESS Guest Investigator Proposal Cycle 4 template
%% V1.0
%% 2017-08-04
%% V1.1
%% 2019-02-07
%% V1.2
%% 2019-10-27
%% V1.3
%% 2020-11-02

%%%%%%%%%%%%%%%%%%%%%%%%%%%
%%%%% DOCUMENT FORMAT %%%%%
%%%%%%%%%%%%%%%%%%%%%%%%%%%

%% The default font was chosen to be easily readable while allowing
%% sufficient material to be included.

%% Please note that the proposal will be printed on US Letter size paper,
%% 8.5 in x 11 in, and that formatting the text for other sizes will
%% generally cause layout problems and may result in text being cut
%% off near the edges. PLEASE DO NOT CHANGE THE 'LETTERPAPER' OPTION
%% IN THE DOCUMENTCLASS COMMAND.

%%%%%%%%%%%%%%%%%%%%%%%%%%%%%%%%%%%%%%%%%%%%%%
%%%%% Default format: 12pt single column %%%%%
%%%%%%%%%%%%%%%%%%%%%%%%%%%%%%%%%%%%%%%%%%%%%%


%% Minimum margin size is 1 inch from top, bottom, and sides.
%% Font size: see NASA Guidebook for Proposers 
%%(https://prod.nais.nasa.gov/pub/pub_library/srba/documents/2020_edition_Proposers_Guidebook.pdf).

\documentclass[letterpaper,12pt]{article}

%%%%%%%%%%%%%%%%%%%%%%%%%%%%%%%%%%
%%%%% HOW TO INCLUDE FIGURES %%%%%
%%%%%%%%%%%%%%%%%%%%%%%%%%%%%%%%%%

%% Please see the ``Included packages'' section below.

%%%%%%%%%%%%%%%%%%%%%%%%%%%%%
%%%%% Included packages %%%%%
%%%%%%%%%%%%%%%%%%%%%%%%%%%%%

\usepackage{graphics,graphicx}
\usepackage[colorlinks]{hyperref}
\hypersetup{urlcolor=blue}
\usepackage{cleveref}

%% Feel free to modify the included packages list to use your
%% favorite packages. 

%% In the graphics and graphicx packages, Postscript and eps figures
%% can be included using the \includegraphics command. The graphics
%% package is part of standard LaTeX2e and provides a basic way of including a
%% figure. The graphicx package is not standard, but extends the
%% \includegraphics command to make it more user-friendly. If graphicx
%% is not available on your system please remove it from the list of
%% included packages above.  

%% Syntax:
%% In the graphics package:
%%
%% \begin{figure}
%% \includegraphics[llx,lly][urx,ury]{file}
%% \end{figure}
%%
%% where ll denotes 'lower left' and ur 'upper right' and the x and y
%% values are the coordinates of the PostScript bounding box in
%% points. There are 72 points in an inch.
%%
%% In the graphicx package:
%% 
%% \begin{figure}
%% \includegraphics[key=val,key=val,...]{file}
%% \end{figure}
%%
%% where some of the useful keys are: angle, width, height,
%% keepaspectratio (='true' or 'false') and scale. Bounding box values
%% can be given as [bb=llx lly urx ury].
%%
%% In either case you have to use LaTeX figure placement commands to
%% position the figure on the page; \includegraphics will not do
%% that. Both these commands also have other options that are listed
%% in the LaTeX manual (for the graphics package) and in 'The LaTeX
%% Graphics Companion' (for the graphicx package).



%%%%%%%%%%%%%%%%%%%%%%%%%%%
%%%%% Page dimensions %%%%%
%%%%%%%%%%%%%%%%%%%%%%%%%%%

\setlength{\textwidth}{6.5in} 
\setlength{\textheight}{9in}
\setlength{\topmargin}{-0.0625in} 
\setlength{\oddsidemargin}{0in}
\setlength{\evensidemargin}{0in} 
\setlength{\headheight}{0in}
\setlength{\headsep}{0in} 
\setlength{\hoffset}{0in}
\setlength{\voffset}{0in}



%%%%%%%%%%%%%%%%%%%%%%%%%%%%%%%%%%
%%%%% Section heading format %%%%%
%%%%%%%%%%%%%%%%%%%%%%%%%%%%%%%%%%

\makeatletter
\renewcommand{\section}{\@startsection%
{section}{1}{0mm}{-\baselineskip}%
{0.5\baselineskip}{\normalfont\Large\bfseries}}%
\makeatother

%%%%%%%%%%%%%%%%%%%%%%%%%%%%%%%%%%%%%
%%%%% Some Useful Abbreviations %%%%% 
%%%%%%%%%%%%%%%%%%%%%%%%%%%%%%%%%%%%%
\newcommand{\tess}{{\it TESS}}
\newcommand{\jwst}{{\it JWST}}
\newcommand{\kepler}{{\it Kepler}}
\newcommand{\ktwo}{{K2}}
\newcommand{\hst}{{\it HST}}
\newcommand{\swift}{{\it Swift}}
\newcommand{\integral}{{\it INTEGRAL}}
\newcommand{\nustar}{{\it NuSTAR}}
\newcommand{\fermi}{{\it Fermi}}
\newcommand{\ms}{$M_{\odot}$}
\newcommand{\rs}{$R_{\odot}$}
\newcommand{\ls}{$L_{\odot}$}
\newcommand{\re}{$R_{\oplus}$}
\newcommand{\me}{$M_{\oplus}$}
\newcommand{\kms}{km~s$^{-1}$}
\newcommand{\fluxcgs}{ergs~s$^{-1}$~cm$^{-2}$}
\newcommand{\lumcgs}{ergs~s$^{-1}$}
\newcommand{\rj}{$R_{\textrm{\scriptsize Jup}}$}
\newcommand{\mj}{$M_{\textrm{\scriptsize Jup}}$}
\newcommand{\ms}{m~s$^{-1}$}


%%%%%%%%%%%%%%%%%%%%%%%%%%%%%
%%%%% Start of document %%%%% 
%%%%%%%%%%%%%%%%%%%%%%%%%%%%%

\begin{document}
\pagestyle{plain}
\pagenumbering{arabic}


 
%%%%%%%%%%%%%%%%%%%%%%%%%%%%%
%%%%% Title of proposal %%%%% 
%%%%%%%%%%%%%%%%%%%%%%%%%%%%%

\begin{center} 
\bfseries\uppercase{%
%%
%% ENTER TITLE OF PROPOSAL BELOW THIS LINE
REPLACE THIS LINE WITH YOUR PROPOSAL TITLE
%%
%%
}
\end{center}



%%%%%%%%%%%%%%%%%%%%%%%%%%%%%%%%%%%%%%%%%
%%%%% Body of science justification %%%%%
%%%%% and technical feasibility     %%%%%
%%%%%%%%%%%%%%%%%%%%%%%%%%%%%%%%%%%%%%%%%


%%%%%%%%%%%%%%%%%%%%%%%%%%%%%%%%%%%%%%%%%
%%%%%%%%%%%%%%%%%%%%%%%%%%%%%%%%%%%%%%%%%
%%%%% The text below should be commented out before submitting your proposal %%%%% 
\noindent{The recommended sections for a \tess\ GI proposal are shown below. Feel free to change section 
headings as necessary, but this is the suggested minimal information that should be included in the proposal. 
This Science/Technical section of the proposal is limited to 4 pages for small programs, and 5 pages for large. Figures are included in these page limits, but references and a (sample) target table are not included in these page limits.\\

\noindent Note that the Phase-1 proposal review will be dual-anonymous  and follow the guidelines listed below: 
\begin{itemize}
    \item Proposals should eliminate language that identifies the proposers or institution, as discussed in the \href{https://science.nasa.gov/researchers/dual-anonymous-peer-review}{Guidelines for Anonymous Proposals}.
    \item PIs are required to upload a one-page Team Expertise (insert link) PDF through a separate upload when submitting the science justification into ARK/RPS.
    \item Proposals that do not follow these dual-anonymous guidelines may be returned without review.
\end{itemize}
 } 
%%%%% The text above should be commented out before submitting your proposal %%%%% 
%%%%%%%%%%%%%%%%%%%%%%%%%%%%%%%%%%%%%%%%%
%%%%%%%%%%%%%%%%%%%%%%%%%%%%%%%%%%%%%%%%%


\section{Introduction}

Summarize the problem being addressed and give an overview of how your investigation will help. 
Why \tess, why now?

\section{Scientific Justification}

Provide text and figures that justify the scientific need for new \tess\ observations and analyses here. When applicable, justify your choice of new 2~min or 20~s cadence observations. 
If you will be making use of the 10~min FFIs for your research, make it clear why the \tess\ FFI data are suitable for your science.
If your program includes theoretical, simulation, or ground-based observing components, describe why these efforts are critical.

%% Trigger Criteria section:
%% comment this section out on proposals not asking for
%% Target of Opportunity (ToO) observations.

\section{({\it Only} For ToO Proposals) Trigger Criteria}

If the proposed investigation includes Targets of Opportunity (ToO's), describe also the circumstances 
under which a ToO is triggered, an estimated duration of the event(s), and an estimated probability for 
triggering the observations. Also discuss the potential science impact imposed by the delay in upload 
of the event due to \tess\ orbit/uplink constraints.

%% Technical Justification for Joint Facilities section
%% comment this section out on proposals not asking for
%% joint time

\section{({\it Only} For Joint \swift\ Proposals) Need for Joint \swift\ Observations}
Justify why you are requesting joint observations with \swift. \\

\noindent Note that \tess\ GI funding is available to successful U.S.-based investigators who request Swift observing time through the \tess\ GI process. No funds will be awarded from the \swift\ project for joint investigations proposed to this \tess\ program element. 

\section{Analysis Plan}

Discuss how you plan to analyze the \tess\ data (or for ground-based observing programs, the data collected). This includes the development of software tools.

\section{Technical Feasibility}

Provide text and figures showing that the proposed \tess\ investigations are feasible; consider the \tess\ survey strategy, target observability, and required signal-to-noise, etc. The \tess\ Science Support Center (\href{https://heasarc.gsfc.nasa.gov/docs/tess/}{TSSC}) makes several tools available to help estimate these quantities. For ground-based observing focused programs, a description of the resources that will be used should be described here.

\section{Expected Impact}

Summarize the expected science return of the proposed investigations and the expected benefit to the community, including new data products and software tools to be made publicly available.

\section{Work Plan}

Provide a brief (1 paragraph) anonymous work plan that provides details on how the proposed effort will be carried out, including the allocation of effort amongst investigators. For example: ``Co-I \#1 will extract the light curves. The PI will mentor a graduate student to model the light curves. Co-I \#2 will lead the collection of ground-based observations.'' 
\\

\noindent All proposals requesting funds must also provide upon submission a bottom-line proposed budget number in the provided field of the ARK submission form; this number should not be included in the body of the proposal.


\section{References}

List of references. References {\it are {\bf not} included} when considering the
proposal page limit. References in the text should be in the number format, and in the references list as:

[1] Person A, Person B, Person C, et al., 2016, ApJ 200, 231, 2\\
[2] Person D \& Person E, 1912, Nature 495, 452


\section{Target Table}

When necessary to justify your proposal, provide a list of targets using the below example as a template for format. This target table is designed to aid reviewers and need only provide a representative sample of the complete target list uploaded to RPS. Full target tables should be submitted electronically with the Phase-1 proposal. Please limit any target table included here to only 1 page. The table is not included in the page limit of the Science/Technical section. 


\begin{center}
\begin{tabular}{ | c | c | c | c | c | c | c | }
\hline
TIC ID          &      Common      &     RA             &      Dec          &      TESS       &       Obj.        &      Comments \\       
                    &      Name           &     (deg)          &      (deg)        &      mag         &       Type       &                         \\     
\hline
\hline
388857263  &  Prox Cen           &  217.428793  &  -62.679592  &  7.36             &    M Dwarf    & 2 min cad., RV planet \\ \hline
353622691  &  BL Lac               &   330.6803807    &   42.2777717    &   13.1  &   AGN            &    20 s cad.                                 \\ \hline
                    &                           &                       &                      &                      &                     &                                     \\ \hline
                    &                           &                       &                      &                      &                     &                                     \\ \hline
                    &                           &                       &                      &                      &                     &                                     \\ \hline
\end{tabular}
\end{center}   

%%%%%%%%%%%%%%%%%%%%%%%%%%%
%%%%% End of document %%%%%
%%%%%%%%%%%%%%%%%%%%%%%%%%%

\end{document}

