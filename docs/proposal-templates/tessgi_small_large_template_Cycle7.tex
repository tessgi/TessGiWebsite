%% LaTeX template for the science justification & technical
%% feasibility to be submitted as part of a TESS Guest Investigator
%% Program proposal. This template is based on the proposal template
%% used by the NuSTAR mission.
%%
%% TESS Guest Investigator Proposal Cycle 7 template
%% V1.0
%% 2017-08-04
%% V1.1
%% 2019-02-07
%% V1.2
%% 2019-10-27
%% V1.3
%% 2020-11-02
%% V1.4
%% 2021-12-03
%% V1.5
%% 2024-01-25

%%%%%%%%%%%%%%%%%%%%%%%%%%%
%%%%% DOCUMENT FORMAT %%%%%
%%%%%%%%%%%%%%%%%%%%%%%%%%%

%% The default font was chosen to be easily readable while allowing
%% sufficient material to be included.

%% Please note that the proposal will be printed on US Letter size paper,
%% 8.5 in x 11 in, and that formatting the text for other sizes will
%% generally cause layout problems and may result in text being cut
%% off near the edges. PLEASE DO NOT CHANGE THE 'LETTERPAPER' OPTION
%% IN THE DOCUMENTCLASS COMMAND.

%%%%%%%%%%%%%%%%%%%%%%%%%%%%%%%%%%%%%%%%%%%%%%
%%%%% Default format: 12pt single column %%%%%
%%%%%%%%%%%%%%%%%%%%%%%%%%%%%%%%%%%%%%%%%%%%%%


%% Minimum margin size is 1 inch from top, bottom, and sides.
%% Font size: see NASA Guidebook for Proposers 
%%(https://prod.nais.nasa.gov/pub/pub_library/srba/documents/2020_edition_Proposers_Guidebook.pdf).

\documentclass[letterpaper,12pt]{article}

%%%%%%%%%%%%%%%%%%%%%%%%%%%%%%%%%%
%%%%% HOW TO INCLUDE FIGURES %%%%%
%%%%%%%%%%%%%%%%%%%%%%%%%%%%%%%%%%

%% Please see the ``Included packages'' section below.

%%%%%%%%%%%%%%%%%%%%%%%%%%%%%
%%%%% Included packages %%%%%
%%%%%%%%%%%%%%%%%%%%%%%%%%%%%

\usepackage{graphics,graphicx}
\usepackage[colorlinks]{hyperref}
\hypersetup{urlcolor=blue}
\usepackage{cleveref}

%% Feel free to modify the included packages list to use your
%% favorite packages. 

%% In the graphics and graphicx packages, Postscript and eps figures
%% can be included using the \includegraphics command. The graphics
%% package is part of standard LaTeX2e and provides a basic way of including a
%% figure. The graphicx package is not standard, but extends the
%% \includegraphics command to make it more user-friendly. If graphicx
%% is not available on your system please remove it from the list of
%% included packages above.  

%% Syntax:
%% In the graphics package:
%%
%% \begin{figure}
%% \includegraphics[llx,lly][urx,ury]{file}
%% \end{figure}
%%
%% where ll denotes 'lower left' and ur 'upper right' and the x and y
%% values are the coordinates of the PostScript bounding box in
%% points. There are 72 points in an inch.
%%
%% In the graphicx package:
%% 
%% \begin{figure}
%% \includegraphics[key=val,key=val,...]{file}
%% \end{figure}
%%
%% where some of the useful keys are: angle, width, height,
%% keepaspectratio (='true' or 'false') and scale. Bounding box values
%% can be given as [bb=llx lly urx ury].
%%
%% In either case you have to use LaTeX figure placement commands to
%% position the figure on the page; \includegraphics will not do
%% that. Both these commands also have other options that are listed
%% in the LaTeX manual (for the graphics package) and in 'The LaTeX
%% Graphics Companion' (for the graphicx package).



%%%%%%%%%%%%%%%%%%%%%%%%%%%
%%%%% Page dimensions %%%%%
%%%%%%%%%%%%%%%%%%%%%%%%%%%

\setlength{\textwidth}{6.5in} 
\setlength{\textheight}{9in}
\setlength{\topmargin}{-0.0625in} 
\setlength{\oddsidemargin}{0in}
\setlength{\evensidemargin}{0in} 
\setlength{\headheight}{0in}
\setlength{\headsep}{0in} 
\setlength{\hoffset}{0in}
\setlength{\voffset}{0in}



%%%%%%%%%%%%%%%%%%%%%%%%%%%%%%%%%%
%%%%% Section heading format %%%%%
%%%%%%%%%%%%%%%%%%%%%%%%%%%%%%%%%%

\makeatletter
\renewcommand{\section}{\@startsection%
{section}{1}{0mm}{-\baselineskip}%
{0.5\baselineskip}{\normalfont\Large\bfseries}}%
\makeatother

%%%%%%%%%%%%%%%%%%%%%%%%%%%%%%%%%%%%%
%%%%% Some Useful Abbreviations %%%%% 
%%%%%%%%%%%%%%%%%%%%%%%%%%%%%%%%%%%%%
\newcommand{\tess}{{\it TESS}}
\newcommand{\jwst}{{\it JWST}}
\newcommand{\kepler}{{\it Kepler}}
\newcommand{\ktwo}{{K2}}
\newcommand{\hst}{{\it HST}}
\newcommand{\swift}{{\it Swift}}
\newcommand{\integral}{{\it INTEGRAL}}
\newcommand{\nustar}{{\it NuSTAR}}
\newcommand{\fermi}{{\it Fermi}}
\newcommand{\nicer}{{\it NICER}}
\newcommand{\ms}{$M_{\odot}$}
\newcommand{\rs}{$R_{\odot}$}
\newcommand{\ls}{$L_{\odot}$}
\newcommand{\re}{$R_{\oplus}$}
\newcommand{\me}{$M_{\oplus}$}
\newcommand{\kms}{km~s$^{-1}$}
\newcommand{\fluxcgs}{ergs~s$^{-1}$~cm$^{-2}$}
\newcommand{\lumcgs}{ergs~s$^{-1}$}
\newcommand{\rj}{$R_{\textrm{\scriptsize Jup}}$}
\newcommand{\mj}{$M_{\textrm{\scriptsize Jup}}$}
\newcommand{\msec}{m~s$^{-1}$}


%%%%%%%%%%%%%%%%%%%%%%%%%%%%%
%%%%% Start of document %%%%% 
%%%%%%%%%%%%%%%%%%%%%%%%%%%%%

\begin{document}
\pagestyle{plain}
\pagenumbering{arabic}


 
%%%%%%%%%%%%%%%%%%%%%%%%%%%%%
%%%%% Title of proposal %%%%% 
%%%%%%%%%%%%%%%%%%%%%%%%%%%%%

\begin{center} 
\bfseries\uppercase{%
%%
%% ENTER TITLE OF PROPOSAL BELOW THIS LINE
REPLACE THIS LINE WITH YOUR PROPOSAL TITLE
%%
%%
}
\end{center}



%%%%%%%%%%%%%%%%%%%%%%%%%%%%%%%%%%%%%%%%%
%%%%% Body of science justification %%%%%
%%%%% and technical feasibility     %%%%%
%%%%%%%%%%%%%%%%%%%%%%%%%%%%%%%%%%%%%%%%%


%%%%%%%%%%%%%%%%%%%%%%%%%%%%%%%%%%%%%%%%%
%%%%%%%%%%%%%%%%%%%%%%%%%%%%%%%%%%%%%%%%%
%%%%% The text below should be commented out before submitting your proposal %%%%% 
\noindent{The recommended sections for a \tess\ GI proposal are shown below. Feel free to change section 
headings as necessary, but this is the suggested minimal information that should be included in the proposal. 
This Science/Technical section of the proposal is limited to 4 pages for small programs, and 6 pages for large programs. Figures are included in these page limits, but references and a (sample) target table are not included in these page limits.\\

\noindent Note that the Phase-1 proposal review will be dual-anonymous  and follow the guidelines listed below: 
\begin{itemize}
    \item Proposals should eliminate language that identifies the proposers or institution, as discussed in the \href{https://science.nasa.gov/researchers/dual-anonymous-peer-review}{Guidelines for Anonymous Proposals}.
    \item PIs are required to upload a one-page \href{https://heasarc.gsfc.nasa.gov/docs/tess/docs/proposal-templates/tessgi\_teamexpertise\_template\_cycle6.tex}{Team Expertise} PDF through a separate upload when submitting the science justification into ARK/RPS.
    \item Proposals that do not follow these dual-anonymous guidelines may be returned without review.
\end{itemize}
 } 
%%%%% The text above should be commented out before submitting your proposal %%%%% 
%%%%%%%%%%%%%%%%%%%%%%%%%%%%%%%%%%%%%%%%%
%%%%%%%%%%%%%%%%%%%%%%%%%%%%%%%%%%%%%%%%%


\section{Introduction}

Summarize the problem being addressed and give an overview of how your investigation will help. 
Why \tess, why now?

\section{Scientific Justification}

Provide text and figures that justify the scientific need for \tess\ observations and analyses here. When applicable, justify your choice of new 2~min or 20~s cadence observations. 
If you will be making use of the 200~s FFIs for your research, make it clear why the \tess\ FFI data are suitable for your science.
If your program includes theoretical, simulation, or ground-based observing components, describe why these efforts are critical.

%% Trigger Criteria section:
%% comment this section out on proposals not asking for
%% Target of Opportunity (ToO) observations.

\section{({\it Only} For ToO Proposals) Trigger Criteria}

If the proposed investigation includes Targets of Opportunity (ToO's), describe also the circumstances 
under which a ToO is triggered, an estimated duration of the event(s), and an estimated probability for 
triggering the observations. Also discuss the potential science impact imposed by the delay in upload 
of the event due to \tess\ orbit/uplink constraints.

%% Technical Justification for Joint Facilities section
%% comment this section out on proposals not asking for
%% joint time

\section{({\it Only} For Joint \swift\ and/or  \nicer Proposals) Need for Joint \swift\ and/or \nicer Observations}

Proposals to this joint program must clearly justify the need for \swift/\nicer ~data and the amount of \swift/\nicer ~time needed to achieve the science goals.\\

These proposals must also present a defined plan for analysis of both the TESS and \swift/\nicer ~data.\\

\noindent Note that \tess\ GI funding is available to successful U.S.-based investigators who request \swift\ and/or \nicer ~observing time through the \tess\ GI process. No funds will be awarded from the \swift\ and/or \nicer\ project for joint investigations proposed to this \tess\ program element. 

\section{Analysis Plan}

Discuss how you plan to analyze the \tess\ data (or for ground-based observing programs, the data collected). This includes the development of software tools.

\section{Technical Feasibility}

Provide text and figures showing that the proposed \tess\ investigations are feasible; consider the \tess\ survey strategy, target observability, and required signal-to-noise, etc. The \tess\ Science Support Center (\href{https://heasarc.gsfc.nasa.gov/docs/tess/}{TSSC}) makes several tools available to help estimate these quantities. For ground-based observing focused programs, a description of the resources that will be used should be described here.

\section{Expected Impact}

Summarize the expected science return of the proposed investigations and the expected benefit to the community, including new data products and software tools to be made publicly available.

\section{Work Plan}

Provide a brief (minimum 1 paragraph) anonymous work plan that provides details on how the proposed effort will be carried out, including the allocation of effort amongst investigators. For example: ``Co-I \#1 will extract the light curves. The PI will mentor a graduate student to model the light curves. Co-I \#2 will lead the collection of ground-based observations.'' 
\\

\noindent All proposals requesting funds must also provide upon submission a bottom-line proposed budget number in the provided field of the ARK submission form; this number should not be included in the body of the proposal.


\section{References}

List of references. References {\it are {\bf not} included} when considering the
proposal page limit. References in the text should be in the number format, and in the references list as:

[1] Person A, Person B, Person C, et al., 2016, ApJ 200, 231, 2\\

[2] Person D \& Person E, 1912, Nature 495, 452


\section{Target Table}

When necessary to justify your proposal, provide a list of targets using the below example as a template for format. This target table is designed to aid reviewers and need only provide a representative sample of the complete target list uploaded to RPS. Full target tables should be submitted electronically with the Phase-1 proposal. Please limit any target table included here to only 1 page. The table is not included in the page limit of the Science/Technical section. 


\begin{center}
\begin{tabular}{ | c | c | c | c | c | c | c | }
\hline
TIC ID          &      Common      &     RA             &      Dec          &      TESS       &       Obj.        &      Comments \\       
                    &      Name           &     (deg)          &      (deg)        &      mag         &       Type       &                         \\     
\hline
\hline
388857263  &  Prox Cen           &  217.428793  &  -62.679592  &  7.36             &    M Dwarf    & 2 min cad., RV planet \\ \hline
353622691  &  BL Lac               &   330.6803807    &   42.2777717    &   13.1  &   AGN            &    20 s cad.                                 \\ \hline
                    &                           &                       &                      &                      &                     &                                     \\ \hline
                    &                           &                       &                      &                      &                     &                                     \\ \hline
                    &                           &                       &                      &                      &                     &                                     \\ \hline
\end{tabular}
\end{center}   

\newpage
%%%%%%%%%%%%%%%%%%%%%%%%%%%%%%%%%%%%%%%%%%%%%%%%%%%%%%%%%%%%%%%%%%%%%
%% Open Science and Data Management plan  %%%%%%%%%%%%%%%%%%%%%%%%%%%
%% This should be no more than two pages %%%%%%%%%%%%%%%%%%%%%%%%%%%%
%% This plan does not count against the science justification limit %
%%%%%%%%%%%%%%%%%%%%%%%%%%%%%%%%%%%%%%%%%%%%%%%%%%%%%%%%%%%%%%%%%%%%%

\begin{center} 
\bfseries\uppercase{
%%
Open Science \& Data Management Plan (OSDMP)
%%
%%
}
\end{center}

%%%%%%%%%%%%%%%%%%%%%%%%%%%%%%%%%%%%%%%%%
%%%%%%%%%%%%%%%%%%%%%%%%%%%%%%%%%%%%%%%%%
%%%%% The text below should be commented out before submitting your proposal %%%%% 

\noindent TESS GI proposals must provide \href{https://science.nasa.gov/researchers/open-science/science-information-policy_faq/}{Open Science and Data Management Plan (OSDMP)}. In the OSDMP proposers must clearly describe the plans to make any new software, higher level data
products and/or supporting data publicly available. Software developed with TESS GI
funds must add value to the TESS science community, be freely available, and have the
source code openly accessible. Proposals that would create software must discuss in their OSDMP
what practices they will follow to develop any tooling. This includes how tools will be
distributed, version controlled, tested, and documented. 

The OSDMP can be no more than 2-pages long, and \textit{does not} count against the page limit for Scientific/Technical/Management. Proposers should refer to the following documents when preparing their OSDMPs:\\

\textbf{Applicable Policies and Requirements:}
\begin{itemize}
\item \href{https://smd-cms.nasa.gov/wp-content/uploads/2023/05/NASA_Plan_for_increasing_access_to_results_of_federally_funded_research1.pdf}{NASA Plan for Increasing Access to Results of Federally Funded Research} 
\item \href{https://smd-cms.nasa.gov/wp-content/uploads/2023/06/SDMWG_Full_Document_v3.pdf}{SMD Strategy for Data Management and Computing for Ground Breaking Science 2019-2024}
\item \href{https://science.nasa.gov/researchers/science-data/science-information-policy/}{SMD Information Policy (SPD-41A)} 
\item \href{https://nspires.nasaprs.com/external/viewrepositorydocument/cmdocumentid=912462/solicitationId=%7bD6BA6770-6C78-B231-74CC-4B8FE91E5AD4%7d/viewSolicitationDocument=1/D.01%20Astro%20Overview%20OSDMP%20clarify%20051523.pdf}{ROSES Section 1.2 of D.1 The Astrophysics Research Program Overview}
\end{itemize}

\textbf{Additional guidance:}
\begin{itemize}
    \item \href{https://science.nasa.gov/researchers/sara/faqs/osdmp/}{ROSES Open Science and Data Management Plan} 
    \item  \href{https://science.nasa.gov/oss-guidance}{SMD Open-Source Science Guidance}
    \item \href{https://github.com/nasa/smd-open-science-guidelines}{SMD Open Science Guidelines GitHub}
    \item \href{https://science.nasa.gov/researchers/open-science/science-information-policy_faq/}{SMD Information Policy FAQ}
  
\end{itemize}

\noindent If there are costs associated with performing the proposed OSDMP tasks, those costs must be accounted for in the proposal budget and/or budget justification. 

Italicized text below is included for explanatory purposes throughout this template and should be omitted from the OSDMP. This template provides one example of the format and contents of an OSDMP.

Questions about OSDMPs can be sent to SARA@nasa.gov or the program scientist Hannah Jang-Condell (hannah.jang-condell@nasa.gov).

%%%%%%%%%%%%%%%%%%%%%%%%%%%%%%%%%%%%%%%%%
%%%%%%%%%%%%%%%%%%%%%%%%%%%%%%%%%%%%%%%%%
%%%%% The text above should be commented out before submitting your proposal %%%%% 

%%%%%%%%%%%%%%%%%%%%%%%%%%%%%%%%%%%%%%%%%%%
%% Template for the OSDMP %%%%%%%%%%%%%%%%% 
%%%%%%%%%%%%%%%%%%%%%%%%%%%%%%%%%%%%%%%%%%%

\section{Data Management Plan}

A data management plan is required for all SMD-funded activities that are expected to produce scientific data. Here it is incorporated into the broader OSDMP. Follow any specific requirements for the data management plan that are provided by the funding solicitation or relevant SMD Division. At a minimum, the DMP includes the following elements:

\subsection{Expected data types, formats, volumes, and standards}

Describe the data expected to be produced from the proposed activities. Include the types of data to be produced, the approximate amount of each data type expected, the machine-readable format of the data, data file format, and any applicable standards for the data or associated metadata.

\subsection{Repositories and timeline for sharing data}

Specify the repository(ies) that will be used to archive and provide public access to data and metadata arising from the activities and the schedule for making data publicly available. Include a description of how data will be archived to enable long-term preservation.

\subsection{Description of data types that are exempt from data sharing requirements}

Specify data types that are excluded from requirements to make the data publicly available and cite the relevant laws, regulations, or policies that generate the exclusion.

\section{Software Management}

A software management plan is required for all SMD-funded activities that are expected to produce software. Here it is incorporated into the broader OSDMP. Follow any specific requirements for the software management plan that are provided by the funding solicitation or SMD Division. If the activity is not expected to produce software, include a statement such as: “No software development is anticipated for this effort. If software is created, it will be made publicly available to the extent legally permitted per the Scientific Information Policy for the Science Mission Directorate.”

\subsection{Expected software types}

Describe the software expected to be produced from the proposed activities, including types of software to be produced, how the software will be developed, and the addition of new features or updates to existing software. This can include the platforms used for development, project management, and community-based best practices to be included such as documentation, testing, dependencies, and versioning.

\subsection{Repositories and timeline for sharing software}

Specify the repository(ies) that will be used to archive software arising from the activities and the schedule for making software publicly available. This should include the license under which the software will be made available. If there are no other restrictions, the software should be released under a permissive license.

\subsection{Description of software that are exempt from software sharing requirements}

Specify software types that are excluded from requirements to make the software publicly available and cite the relevant laws, regulations, or policies that generate the exclusion.

\section{Open Science Plan}

\subsection{Publication Sharing}

Describe the types of publications that are expected to be produced from the activities (e.g., peer reviewed manuscripts, technical reports, conference materials, and books). Outline the methods expected to be used to make the publications publicly available, which will likely include options listed under ‘How to Share Publications’ in the SMD Open-Source Science Guidance. This may include posting manuscripts to community-appropriate preprint servers, making accepted manuscripts publicly available in NASA's STI Repository, or publishing manuscripts as Open Access in reputable journals. Note that costs for Open Access publishing may be included in proposal budgets.

\subsection{Other Open Science Activities}

Optionally, the OSDMP may include a description of additional open science activities associated with the project (if not described elsewhere in a proposal). This may include: holding scientific workshops and meetings openly to enable broad participation, providing project personnel with open science training or enablement, implementing practices that support the inclusion of broad, diverse communities in the scientific process as close to the start of research activities as possible if not described elsewhere, and contributions to or involvement in open-science communities.

\section{Roles and Responsibilities}

Specify the project personnel who will ensure the implementation of the OSDMP. This may be its own section or integrated into the sections above.

%%%%%%%%%%%%%%%%%%%%%%%%%%%
%%%%% End of document %%%%%
%%%%%%%%%%%%%%%%%%%%%%%%%%%

\end{document}

